%% Generated by Sphinx.
\def\sphinxdocclass{report}
\documentclass[letterpaper,10pt,english]{sphinxmanual}
\ifdefined\pdfpxdimen
   \let\sphinxpxdimen\pdfpxdimen\else\newdimen\sphinxpxdimen
\fi \sphinxpxdimen=.75bp\relax
\ifdefined\pdfimageresolution
    \pdfimageresolution= \numexpr \dimexpr1in\relax/\sphinxpxdimen\relax
\fi
%% let collapsible pdf bookmarks panel have high depth per default
\PassOptionsToPackage{bookmarksdepth=5}{hyperref}

\PassOptionsToPackage{booktabs}{sphinx}
\PassOptionsToPackage{colorrows}{sphinx}

\PassOptionsToPackage{warn}{textcomp}
\usepackage[utf8]{inputenc}
\ifdefined\DeclareUnicodeCharacter
% support both utf8 and utf8x syntaxes
  \ifdefined\DeclareUnicodeCharacterAsOptional
    \def\sphinxDUC#1{\DeclareUnicodeCharacter{"#1}}
  \else
    \let\sphinxDUC\DeclareUnicodeCharacter
  \fi
  \sphinxDUC{00A0}{\nobreakspace}
  \sphinxDUC{2500}{\sphinxunichar{2500}}
  \sphinxDUC{2502}{\sphinxunichar{2502}}
  \sphinxDUC{2514}{\sphinxunichar{2514}}
  \sphinxDUC{251C}{\sphinxunichar{251C}}
  \sphinxDUC{2572}{\textbackslash}
\fi
\usepackage{cmap}
\usepackage[T1]{fontenc}
\usepackage{amsmath,amssymb,amstext}
\usepackage{babel}



\usepackage{tgtermes}
\usepackage{tgheros}
\renewcommand{\ttdefault}{txtt}



\usepackage[Bjarne]{fncychap}
\usepackage{sphinx}

\fvset{fontsize=auto}
\usepackage{geometry}


% Include hyperref last.
\usepackage{hyperref}
% Fix anchor placement for figures with captions.
\usepackage{hypcap}% it must be loaded after hyperref.
% Set up styles of URL: it should be placed after hyperref.
\urlstyle{same}


\usepackage{sphinxmessages}
\setcounter{tocdepth}{1}



\title{CatchmentCollector Documentation}
\date{Feb 18, 2025}
\release{0.1}
\author{DG Plugins}
\newcommand{\sphinxlogo}{\vbox{}}
\renewcommand{\releasename}{Release}
\makeindex
\begin{document}

\ifdefined\shorthandoff
  \ifnum\catcode`\=\string=\active\shorthandoff{=}\fi
  \ifnum\catcode`\"=\active\shorthandoff{"}\fi
\fi

\pagestyle{empty}
\sphinxmaketitle
\pagestyle{plain}
\sphinxtableofcontents
\pagestyle{normal}
\phantomsection\label{\detokenize{index::doc}}


\sphinxAtStartPar
The purpose of the plugin is to download the characteristic cards of selected catchments from the website: \sphinxhref{http://karty.apgw.gov.pl:4200/informacje}{Jednolite czesci wod} to the folder of a given project.


\chapter{Using the plugin}
\label{\detokenize{index:using-the-plugin}}
\sphinxAtStartPar
To use this plugin simply go to the QGIS plugin managerand install it on your system just like any other QGIS plugin.  You will see a toolbutton in the plugins toolbar with the following image.

\noindent\sphinxincludegraphics{{icon}.png}

\sphinxAtStartPar
Before running the plugin, select the layer with the catchments and select the element of interest (you can use the “select by location..” tool in the vector/research tools menu)

\noindent\sphinxincludegraphics{{layer_and_field_selected}.png}

\sphinxAtStartPar
After clicking this toolbutton the following dialog will appear

\noindent\sphinxincludegraphics{{dialog}.png}

\sphinxAtStartPar
Choose the layer witch Catchment that you want to use (the previously marked layer with the catchment should already be selected) and the field that defines the code field.

\sphinxAtStartPar
Click OK and you will see a message dialog similar to the one below with list of downloaded cards.

\noindent\sphinxincludegraphics{{close_message_list}.png}

\sphinxAtStartPar
The cards will be downloaded to a newly created ‘Safety Data Sheets’ folder in the project folder.

\noindent\sphinxincludegraphics{{New_folder}.png}


\chapter{Indices and tables}
\label{\detokenize{index:indices-and-tables}}\begin{itemize}
\item {} 
\sphinxAtStartPar
\DUrole{xref,std,std-ref}{genindex}

\item {} 
\sphinxAtStartPar
\DUrole{xref,std,std-ref}{modindex}

\item {} 
\sphinxAtStartPar
\DUrole{xref,std,std-ref}{search}

\end{itemize}



\renewcommand{\indexname}{Index}
\printindex
\end{document}